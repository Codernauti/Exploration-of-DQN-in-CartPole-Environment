\section{Background}

%The agent each step t interact with the environment and receive a reward. 

The interaction between the agent and the environment is described  with the
Markov Decision Process (MDP) formalism. A MDP is a tupla $\langle \mathcal{S}, 
\mathcal{A}, \mathcal{P}, \mathcal{R}, \gamma \rangle$ where $\mathcal{S}$ is
the set of states, $\mathcal{A}$ is the set of actions, $\mathcal{P}$ is the 
state transition probability, $\mathcal{R}$ is the reward function (???) and 
$\gamma$ is the discount factor.

The main goal of an agent is to estimate the action value function $Q$ that 
computes how much expected return come from a state $s$ given an action $a$.


Unluckily in a real context the MDP is unknown or its representation is too big
(ex. game Go). To face this issue numerous methods were proposed. They are 
called model free because they didn't need any priori knowledge and they learn 
only from sampling.

One of this famous method is the Q-learning: a model free off policy [Watkins 
1989]. Q-learning learns using two policy, the behavior policy $\mu$ that (???) 
and the target policy $\pi$.


