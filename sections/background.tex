\section{Background}

%The agent each step t interact with the environment and receive a reward. 

The interaction between the agent and the environment is described  with the
Markov Decision Process (MDP) formalism. A MDP is a tupla $\langle \mathcal{S}, 
\mathcal{A}, \mathcal{P}, \mathcal{R}, \gamma \rangle$ where $\mathcal{S}$ is
the set of states, $\mathcal{A}$ is the set of actions, $\mathcal{P}$ is the 
state transition probability, $\mathcal{R}$ is the reward function (???) and 
$\gamma$ is the discount factor.

The main goal of an agent is to interact with the environment maximizing the future rewards. In order to do this it estimates the action value function $Q$ that computes how much expected return come from a state $s$ given an action $a$: $Q(s, a) = \max_\pi E[R_t | s_t = s, a_t = a, \pi]$ where $\pi$ is a policy: a function that maps states to actions.


Unluckily in a real context the MDP is unknown or its representation is too big
(ex. game Go). To face this issue numerous methods were proposed, they are 
called \textit{model free} because they didn't need any prior knowledge and they learn only from sampling.


\subsection{Q-learning}

One method that tackles the previous issue is the \textbf{Q-learning} algorithm: a model free off policy method \cite{Watkins1992}. 
\textit{Off-policy} means that the Q-learning learning use two policy. The first one is the target policy $\pi$, that follows the \textit{greedy} strategy $a = \max_a Q(s, a; \theta)$, that is selecting always the action that will return the maximum reward.
The second one is the \textit{behavior policy} $\mu$ that ensures an exploration of the states space, typically an \textit{$\epsilon$-greedy} strategy. \textit{$\epsilon$-greedy} is a greedy strategy only with probability $1 - \epsilon$ and selects a random action with probability $\epsilon$.

It estimate the action-value function using the following update rule:
\begin{equation}
	Q(s,a) = Q(s, a) + \alpha (r + \gamma \max_{a'} Q(s', a') - Q(s, a))
\end{equation}


\subsection{Double Q-learning}

It is proved that Q-learning overestimates the action-value function 
\cite{NIPS2010_3964}. This is caused by the approximation of the maximal estimator $\max_a Q(S_{t+1}, a)$ that is a bias estimator of $max_a E \{ Q(S_{t+1}, a) \}$. 

