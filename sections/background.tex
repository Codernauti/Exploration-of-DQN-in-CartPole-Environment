\section{Background}

%The agent each step t interact with the environment and receive a reward. 
The interaction between the agent and the environment is described  with the
Markov Decision Process (MDP) formalism. A MDP is a tupla $\langle \mathcal{S}, 
\mathcal{A}, \mathcal{P}, \mathcal{R}, \gamma \rangle$ where $\mathcal{S}$ is
the set of states, $\mathcal{A}$ is the set of actions, $\mathcal{P}$ is the 
state transition probability, $\mathcal{R}$ is the reward function and 
$\gamma \in [0,1]$ is the discount factor determining the agent's horizon.

The main goal of an agent in RL is to interact with the environment maximizing the future rewards. In order to do this it estimates the action value function $Q$ that computes how much expected return come from a state $s$ given an action $a$: $Q^*(s, a) = \max_\pi E[ G_t | s_t = s, a_t = a, \pi]$ where $\pi$ is a policy, a function that maps states to actions and $G_t = R_t + \gamma~R_{t+1} + \gamma^2~R_{t+2} + \dots$ is the discounted return.


Unluckily in a real context the MDP is unknown or its representation is too big
(ex. game Go). To face this issue numerous methods were proposed, they are 
classified as \textit{model-free} methods because they didn't need any prior knowledge and they learn only from sampling.

\subsection{Q-learning}

One method that tackles the previous issue is the \textbf{Q-learning} algorithm: a form of \textit{temporal difference learning} \cite{Sutton:1998:IRL:551283}

Q-learning is defined as a \textit{model-free} \textit{off-policy} method \cite{Watkins1992}. \textit{Off-policy} means that the Q-learning learns using two policy. The first one is the target policy $\pi$, that follows the \textit{greedy} strategy $a = \max_a Q(s, a; \theta)$, that is selecting always the action that will return the maximum reward.
The second one is the \textit{behavior policy} $\mu$ that ensures an exploration of the states space, typically an \textit{$\epsilon$-greedy} strategy. \textit{$\epsilon$-greedy} is a greedy strategy only with probability $1 - \epsilon$ and selects a random action with probability $\epsilon$.

The Q-learning first initialize arbitrarily the $Q(s,a)$ except for the terminal state where $Q$ is initialize to 0.
The agent from the initial state $S_0$ starts the iteration identified by a step $t$. Inside this loop the agent choose an action $A_t$ from $S_t$ (current state) using the \textit{behavior policy}. Then it takes the action $A_t$ and observe from the environment the immediate reward $R_{t+1}$ and the resulting state $S_{t+1}$ with which it estimates the action-value function through the following update rule:
\begin{equation}
	Q(S_t, A_t) = Q(S_t, A_t) + \alpha (Y^Q_t - Q(S_t, A_t))
\end{equation}
Where $\alpha$ is a scalar step size and the target $Y^Q_t$ is defined as
\begin{equation}
	Y^Q_t \equiv R_{t+1} + \gamma \max_{a} Q(S_{t+1}, a)
\end{equation}

Finally $S'$ is assigned to $S$. The iteration goes on until $S$ is a terminal state.

\subsection{Deep Q Network algorithm}
\label{subsec:DQN}

%(What do the authors want to do?)
In the \textit{Playing Atari with Deep Reinforcement Learning}~\cite{DBLP:journals/corr/MnihKSGAWR13} work a new successful result using Q-learning was obtained implementing the first version of the \textbf{Deep Q Network} (\textbf{DQN}) \textbf{algorithm}. This DQN trained with Q-learning learnt successfully to play 7 Atari 2600 games only from high-dimensional sensory input (pixels) given by the \textit{Atari Learning Environment} \cite{ArcadeLearningEnvironment}.

\begin{table}[]
	\centering
	\begin{tabular}{|l|r|}
		\hline
		\multicolumn{1}{|c|}{\textbf{Hyperparameter}} &
		\multicolumn{1}{|c|}{\textbf{Value}}        \\
		\hline
		Discount factor $\gamma$ & 0.99                        \\
		Learning rate $\alpha$   & 0.00025                     \\
		Target network update    & 10,000 steps                \\
		Replay memory dimension  & 1 Million                   \\
		Exploration policy       & From 1 to 0.1 over 1M steps \\
		\hline
	\end{tabular}
	\caption{Hyperparameters used in DQN algorithm and Double DQN algorithm.}
	\label{fig:dqn-hyperparams}
\end{table}

DQN is a multi-layered Convolution Neural Network $Q(s, \cdotp; \theta)$ %(TODO: CITARE L'ARCHITETTURA???), % 
where $\theta$ are the parameters, that approximates the action-value function $Q$. It takes a 84x84x4 of the last four gray-scale frames as input representing the state $s$, and it outputs one predicted $Q$-value for each valid action.  

As the Q-learning, DQN algorithm applies \textit{greedy} as target policy and \textit{$\epsilon$-greedy} as behaviour policy. Initially the parameter $\epsilon$ is set to 0 and it linearly decreases at each step to the minimum 0.01. The main hyperparameters are showed in Table~\ref{fig:dqn-hyperparams}.

%(What were the general issues?)
%(How do they solve them?)
One important issue faced is that the agent has to learn from noisy data highly correlate. In order to break these correlations they successfully applied the \textit{experience replay} technique~\cite{Lin:1992:RLR:168871}. It consists in building a data-set of samples, the \textit{replay memory} ($D$), in which at each step the algorithm stores a new sample $e_t = (s_t, a_t, r_{t+1}, s_{t+1})$ overwriting the most old one. Next the algorithm applies the Q-learning update on samples of experience $(S_t, A_t, R_{t+1}, S_{t+1}) \sim U(D)$ taken uniformly at random from this memory. The Q-learning update at step $t$ uses the following loss:
\begin{equation}
	L_t(\theta_t) = E_{e_t \sim U(D)} % 
		\Big[ \Big( Y^{DQN}_t - Q(S_t, A_t;\theta_t) \Big)^2 \Big]
\end{equation}
With the target define as:
\begin{equation}
Y^{DQN_1}_t \equiv R_{t+1} + \gamma \max_{a} Q(S_{t+1}, a; \theta_t)
\end{equation}

Using \textit{experience replay} implies a greater data efficiency since each sample is potentially used in many weight updates and it avoids oscillations or divergence in the parameters of the network.

%(CUT)
%(What is the result?)
%In the seven games used as testing environment the DQN algorithm opportunely tuned has given state-of-the-art results in six of them.
%(CUT)

%(What do the authors want to do?)
In the article \textit{Human-level control through deep reinforcement learning}~\cite{Mnih2015} the \textbf{DQN algorithm} is improved using a \textit{target network} in addition to \textit{experience replay}.
%(How does the algorithm work?) %
The second version of DQN algorithm applies an \textit{$\epsilon$-greedy} policy with the network $Q(s, \cdotp; \theta)$ (\textit{online network}) and it applies the \textit{greedy} policy with the \emph{different} network $Q(s, \cdotp; \theta')$ (\textit{target network}). The parameters $\theta'$ are updated with a periodically copy of the $\theta$ parameters from the \textit{online network}. The target is now defined as:
\begin{equation}
Y^{DQN_2}_t \equiv R_{t+1} + \gamma \max_{a} Q(S_{t+1}, a; \theta_t')
\end{equation}

%(What is the result?)
\citeauthor{Mnih2015} \shortcite{Mnih2015} demonstrated that the union of \textit{experience replay}, separate \textit{target Q-network} and deep convolutional network architecture are essential for outperforming the best existing reinforcement learning methods on 43 of 57 Atari 2600 games. Furthermore the trained agents was managed to achieve more than 75\% of the human score on 29 games.



\subsection{Double Q-learning}
\label{subsec:DoubleQlearning}

It is known that Q-learning overestimates the action-value function~\cite{NIPS2010_3964}. This is caused by the approximation of the maximal estimator $\max_a Q(S_{t+1}, a)$ that is a bias estimator of $max_a E \{ Q(S_{t+1}, a) \}$, this bias is cumulative at each update. In order to remove this effect the \textbf{Double Q-learning} algorithm was proposed~\cite{NIPS2010_3964}.

The main idea of Double Q-learning is to use a double estimator: keeping two action-value function $Q^A$ and $Q^B$. At each step only one of two function is updated and in the update instead of using the $Q^A(S{t+1}, a^*) = \max_a Q(S_{t+1}, a)$ where $a^* = \max_a Q^A(S_{t+1}, a)$ it is used $Q^B(S_{t+1}, a^*)$.
Van Hasselt (2010) proved mathematically and in small environments that the Double Q-learning is more accurate than Q-learning and sometimes underestimate.
\
\subsection{Double Deep Q Network Algorithm}

% What did the authors demontrate?
%(What were the general issues?)
\textit{Deep Reinforcement Learning with Double Q-learning}~\cite{Hasselt:2016:DRL:3016100.3016191} showed that DQN algorithm suffers from overestimations impacting negatively on the quality of the policy learned. They demonstrate that estimation errors of any kind can induce an upward bias driving the estimate up and away from the true optimal value.

% How did they enhanced the algorithm?
In order to reduce this overestimation \citeauthor{Hasselt:2016:DRL:3016100.3016191} \shortcite{Hasselt:2016:DRL:3016100.3016191} proposed the \textbf{Double Deep Q Network} (\textbf{DDQN}), a DQN adaptation that implements the idea of the Double Q-learning. The only change from the previous DQN~\cite{Mnih2015} was the target $Y_t^{DQN_2}$ replaced with:
\begin{equation}
	Y^{DDQN}_t \equiv R_{t+1} + \gamma Q(S_{t+1}, 
		\arg\max_{a} Q(S_{t+1}, a; \theta_t); \theta_t^-)
\end{equation}

Using the exact same hyperparameters of DQN algorithm the Double DQN obtained clearly better results showing that reducing overestimations can significantly benefit the stability of learning.


