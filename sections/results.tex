\section{Results Discussion}

\subsection{Training}

% Arguments:

From Figure~\ref{fig:q-values} we can see the different performances from the 4 configuration. In order to reduce the noise of the Q-value estimates, as the cited papers did, during the training we take the averages of the Q-values from 64 steps, defined as \textit{epoch}. More epochs computed by a configuration implies more steps per episode and then more reward in general. Hence the \textit{deep} configurations played better than the \textit{shallow}. More details in the evaluation subsection.

A more interesting viewpoint is the overestimation issue. If the reward is nonzero and $\gamma < 1$ as in the \textit{CartPole} environment the return $G_t$ is $\frac{1}{1 - \gamma}$ \cite{Sutton:1998:IRL:551283}. Hence in our case the return with $\gamma = 0.99$ is $100$ then the DQN has to converge to that value.

From Figure~\ref{fig:q-values} we can see that the trends of all the configuration is to converge to 100. As described by \citeauthor{Hasselt:2016:DRL:3016100.3016191} \shortcite{Hasselt:2016:DRL:3016100.3016191} the Double Q-learning reduces the overestimation, in particular for the \textit{Double DQN deep} configuration, the \textit{DQN deep} suffers definitely of overestimation. For \textit{DQN shallow} configurations instead the differences are small, maybe the differences could start in later epochs.

\subsection{Evaluation}

% Theses:
From the results of evaluation sessions we saw that our \textit{DQN deep} is better in general

\begin{figure}
	\centering
	\includegraphics[width=0.48\textwidth]{res/Comparison}
\end{figure}


\begin{figure*}[t]
	\centering
	\subfloat{\includegraphics[width=.48\textwidth]{res/DQN_Shallow}} \quad
	\subfloat{\includegraphics[width=.48\textwidth]{res/DoubleDQN_Shallow}} \\
	\subfloat{\includegraphics[width=.48\textwidth]{res/DQN_Deep}} \quad
	\subfloat{\includegraphics[width=.48\textwidth]{res/DoubleDQN_Deep}} \\
	
	%\subfloat[][\emph{Cascata}.]
	%{\includegraphics[width=.45\textwidth]{Cascata}} \quad
	%\subfloat[][\emph{Salita e discesa}.]
	%{\includegraphics[width=.45\textwidth]{SalitaDiscesa}}
	\caption{The blue lines in the figure represents the median of the Q-value  per epoch (the average Q-value computed every 64 steps) from the executions of 3 seeds for each configuration. The red line shows the maximum and minimum values from the results.}
	\label{fig:q-values}
\end{figure*}